% http://www.miccai2013.org/submission_guideline.html
\documentclass{llncs}

\usepackage{makeidx}
\usepackage[pdftex]{graphicx}

\def\wrt{w.\,r.\,t.}
\def\eg{e.\,g.}
\def\ie{i.\,e.}
\def\Dash{\nobreak\,---\penalty-500\,}

\def\TODO{{\color{red}{TODO}}}


%%% Comments
\usepackage{color}
\newcommand{\TG}[1]{{\color{blue}\textbf{TG: #1}}}
\newcommand{\CD}[1]{{\color{green}\textbf{CD: #1}}}
\newcommand{\IP}[1]{{\color{cyan}\textbf{IP: #1}}}
%%%

\newcommand{\Vec}[1]{\mathbf{#1}}
\newcommand{\Mat}[1]{\mathbf{#1}}

%\Figure{width}{name}{caption}
\newcommand{\Figure}[3]{%
\begin{figure}[htb]
  \centering
  \includegraphics[width=#1]{#2}
  \caption{\label{fig-#2}#3}
\end{figure}}

\begin{document}
%
%\frontmatter          % for the preliminaries
%
\mainmatter              % start of the contributions
%
\title{Complete Real-Time Liver Model including Glisson Capsule, Vascular Network and Parenchyma}
%
\titlerunning{TODO: Title}  % abbreviated title (for running head)
%                                     also used for the TOC unless
%                                     \toctitle is used
%
\author{Anonymous}
%\author{%
%Tom\'a\v{s} Golembiovsk\'y\inst{1,2} \and%
%Igor Peterl\'ik\inst{3} \and%
%Christian Duriez\inst{2} \and%
%Stephan\'e Cotin\inst{2,3}%
%}
%
%\authorrunning{Tom\'a\vs Golembiovsk\'y et al.} % abbreviated author list (for running head)
%
%\institute{%
%Faculty of Informatics, Masaryk University, Brno, Czech Republic\and%
%INRIA Lille -- Nord, France\and%
%IHU, Strasbourg, France%
%}
%
\maketitle

\begin{abstract}
This work provides a complete physical model for the deformations of the liver with its three constituents: parenchyma, vessels and the Glisson's capsule.
Some recent experimental results \cite{Ahn2010} have shown that there is a significant difference of stiffness between the Glisson's capsule and the parenchyma. 
Moreover, it has been observed that the capsule plays an important role for the deformation at the local level \cite{Hollenstein2006}.
However, the Glisson's capsule is very thin and its influence for the global deformations of the liver was never highlighted.
In this work, we propose to use a membrane FEM model for the capsule, that is coupled with a vascularized model of the liver (\CD{based on a coupling ?})
We show that the approach is able to reproduce the deformations observed at the local level.
Additionally, we show, by simulating natural global deformations of the liver observed on real images (due to gravity), that the results have important differences with and without the capsule.

\keywords{liver, Glisson's capsule, vascular network, deformation modeling, membrane}
\end{abstract}

\section{Introduction} 
\IP{I think we should provide some introductory motivation and broader context of the reasarch. I tried this:}

In 2009, nearly 100,000 European citizens died of cirrhosis or liver cancer. 
According to the statistics, nearly 100,000 European citizens die of cirrhosis or liver cancer each year. 
Although new methods such as radio-frequency ablation or cryo-ablation used in interventional radiology 
represent promising improvements, surgery remains the option that offers the foremost success rate against these pathologies. 
Nevertheless, surgery is not always performed due to several limitations, in particular the determination 
of accurate eligibility criteria for the patient. 
In this context, pre-operational planning becomes a crucial task having a significant impact on the treatment. 

Computer-aided physics-based medical simulation has proven to be an extremely useful technique in the area of medical training. 
Nevertheless, whereas generic models are usually required in training simulators, accurate patient-specific modelling
becomes necessary as soon as computer simulation is to be employed in the pre-operative planning. At the same time, 
interactivity of such models  remains an important aspect, requiring real-time simulation which is often difficult to 
achieve given the complexity of soft tissues. 

In case of human liver, the task of real-time accurate modeling is challenging, mainly because of the complex structure 
of this organ composed of three main constituents: \emph{parenchyma}, \emph{vascular networks} and \emph{Glisson's capsule}.
The parenchyma is certainly the most studied component of the liver; today the researchers agree on hyperelastic 
properties of the tissue, for which the mechanical parameters have been reported for example in~\cite{}. 
Several methods have been proposed to model the hyperelastic behaviour at real-time rates, such multiplicative Jacobian decomposition
introduced in~\cite{stephanie2008}.

The mechanical importance of the vascular structures in liver is studied in~\cite{peterlik2012}. It shows that that the 
influence of vessels on the mechanical behaviour of the liver is significant, mainly if large deformations occur. 
Since detailed modeling of the vessels would be extremely costly (mainly because of the thickness of the vessel wall 
being less than 250\,{$\mu$}m), the authors propose a composite model allowing for real-time simulation of entire liver
with two vascular trees corresponding to portal and hepatic veins.

The final constituent represented by Glisson's capsule have been already studied in the literature. 
Extensive experiments on porcine liver has been published in~\cite{Sagar}. The measurements show that although being very 
thin (10--20$\mu m$), the Young's modulus of the capsule (reported as $8.22\pm3.42$\,MPa) exceeds the values for the parenchyma by three orders of magnitude,
suggesting the mechanical influence of the membrane on the liver behaviour.

In~\cite{Hollenstein}, a local influence of the capsule has been measured using a special aspiration device. The study was then repeated 
in vivo on human patients during the operation, confirming the mechanical importance of the membrane~\cite{Ahn,Nava}.
To our best knowledge, on attempt has been made to demonstrate the role of the Glisson's capsule on the global behaviour of 
the liver, i.e. during large deformations of the organ.

The main contribution of the paper is twofold: first, we present an efficient real-time model of the liver, taking into account 
the parenchyma, vascularization and Glisson's capsule. The model is based on three different finite element representations for each constituent,
linked together via mechanical coupling. We show that the model mimics the local experiments descibed in~\cite{Hollenstein}.
 
Second, we use the complete model of the liver to demonstrate the influence of the Glisson's
capsule via simulation: using a specific model of a pig liver built from CT contrast-enhanced data, we show that there is a significant 
difference in the response of the model with and without the capsule in case when the liver undergoes large deformations. 

The paper is organized as follows: \TODO.


%%Medical simulation is expected to play an integral part in many aspects of
%%medical practice. From diagnosis, through planning or training to computer
%%assisted intervention. A good medical model has to be simple yet efficient
%%to be really usable.
%\CD{develop the applications for which it has to be real-time...}
%
%\CD{First... explain the biomechanics of the liver (for instance: 
%\textit{
%The liver is a complex organ composed by three main structures: the parenchyma (soft tissue, that have a hyper-visco-elastic behavior) is coupled with a dense vascular network (tree-like tubular structure with a stiffer behavior) and is surrounded by the Glisson capsule  (a stiff collagenous membrane).
%It illustrates the need of relying on both volume, surface and line finite elements and the need of proposing coupling methods between them.}
% }
%
%\CD{Yet, many work on liver modeling only consider the parenchyma...}
%The liver is usually modeled as a homogeneous isotropic (visco-)elastic
%material \TG{refs}.% It is however non-homogeneous and modeling it as
%%homogeneous necessarily requires complex non-linear material descriptions.
%Recently
%Peterl\'{i}k et al. \cite{Peterlik2012} showed the importance of the vascular
%structures on the liver deformation.
%
%Another often overlooked component is the Glisson's capsule.
%Biomechanical measurements report that the capsule has very small
%thickness (approx. 20 $\mu$m for porcine liver \cite{Umale2011}).
%On the other hand it has in three orders of magnitude higher stiffness than
%the parenchyma.
%Because of this it plays an important role in local
%deformations of the liver \cite{Ahn2010,Hollenstein2006}.
%It was reported that evaluation of the liver as a homogeneous material
%without capsule can lead to overestimation of local deformations up to the factor of 3
%\cite{Hollenstein2006}.
%
%While the influence on local deformation was reported, no experimental
%results show whether the capsule also plays an important role on the global
%deformations of the liver. We propose a composite model that models the
%capsule with membrane FEM elements. We show that the model is able to
%simulate expected local behaviour reported in literature. Further more we
%show that the capsule plays also an important role in the deformation on
%global scale.
%
\TG{summary of following sections.}

%}}}


\section{Model} %{{{

In this section we will describe the construction of our composite model.
It's three main parts are: tetrahedral model for parenchyma, membrane model
for capsule coupled to the parenchyma and beam model of vascular system as
proposed by Peterl\'{i}k et al. \cite{Peterlik2012}.

\subsection{Parenchyma} %{{{

It is known that the parenchyma exhibits viscoelastic behaviour \TG{Cite
something}, we however employ simpler elastic model as we are not so much
interested in time-dependent behaviour but rather in static equilibrium
under certain conditions.
We also rely on the vascularized model of the parenchyma proposed by
Peterl\'{i}k et al. \cite{Peterlik2012}.

The parenchyma is modeled using corotational finite elements. The corotational
formulation allows us to deal with geometrical non-linearities but still
rely on linear stress-strain relationship. Thus rotations and large
deformations are possible. In the corotational formulation the deformation
of each element is expressed in the local frame of reference. There are
several ways of computing the corotational frame for elements, we rely on
the geometrical method proposed in \cite{Nesme2005}.
% NOTE: This description of corotational method is very simplified and could be extended.
The stiffness matrix $\Mat{K}$ can be expressed in terms of the deformation
$u$ and the force as $\Vec{f} = \Mat{K}(\Vec{u}) \Vec{u}$. Where the
components of $\Mat{K}$ depend on the orientation of each element in the
current simulation step.

The vascularization method is based on linear beams
with local frames of reference (in a many aspects similar to the
corotational formulation). As such the model also handles geometric
non-linearities in the deformation. The proposed model assumes no relative
motion of vessels to the parenchyma. Segmented vascular network is
discretized into the set of linear elements that are then projected into
the tetrahedral mesh of the parenchyma. During the simulation forces and
torques resulting from the deformations of vessels by parenchyma are
computed and propagated back onto the tetrahedral mesh.

%}}}

\subsection{Glisson's Capsule} %{{{

The thickness of the Glisson's capsule is relatively small. Umale et al.
\cite{Umale2011} reported for porcine liver values in range of 10-20
$\mu$m.
It is not possible to model such thin structure with classical tetrahedral
elements. Modeling this non-homogeneous volumetric object would require a
dense mesh to avoid numerical instabilities and would significantly
violate the speed requirements imposed on medical simulators.
Instead modeling the capsule with two-dimensional elements that abstract from the
thickness in the third dimension seems
as a natural choice. Mechanics provides this a FE tool in the form of membranes
and shells. Based on the small thickness (compared to the surface area) we also
assume negligible bending forces and propose a model based on membrane
elements. 
To maintain simplicity of the composite model we choose simple triangular
elements with constant strain.

\TG{add a ref. or dive into the definition? \cite{Felippa2003}} \CD{both...}

We again use linear elastic material and employ the corotational formulation
for the triangular elements.

%}}}

\subsection{Coupling Between Capsule and Parenchyma} %{{{

The literature reports high cohesion between capsule and parenchyma.
Based on this
property we may assume no relative motion of the capsule to the parenchyma.
While in general we could use arbitrary surface mesh for a capsule we use
the fact that the parenchyma is modeled by tetrahedral elements which have
triangular faces. Because the boundary of the volumetric mesh is already
triangulated, we only extract the boundary triangles which we use for the
model of capsule.

Using directly the boundary of the tetrahedral mesh does not only solve the
problem of obtaining the surface mesh, but has one more advantage. Nodes
of the triangular mesh overlap with nodes of the tetrahedral mesh and we do
not have to project nodes of triangular mesh onto the tetrahedra. More over,
the stiffness matrices for capsule and parenchyma are easily assembled
together and solved as one system.

\CD{The following may appear as "trivial"... maybe we can remove this part at the end if we need space} 
Without the loss of generality we can assume the tetrahedron consists of
nodes $p_1, p_2, p_3$ and $p_4$ and the boundary triangle has nodes $p_1, p_2$
and $p_3$. We can reorder the degrees of freedom so that the stiffness
matrix $\Mat{K}^t$ for the tetrahedron can be written as:

\begin{equation}
  \Mat{K}^t = \left[\begin{array}{c|c}
      \Mat{K}^t_{1-3,1-3} & \Mat{K}^t_{1-3,4} \\
      \hline
      \Mat{K}^t_{4,1-3} & \Mat{K}^t_{4,4} \\
  \end{array}\right]
\end{equation}

Then the assembled stiffness matrix for the element is:

\begin{equation}
  \Mat{K} = \left[\begin{array}{c|c}
      \Mat{K}^t_{1-3,1-3} & \Mat{K}^t_{1-3,4} \\
      \hline
      \Mat{K}^t_{4,1-3} & \Mat{K}^t_{4,4} \\
  \end{array}\right]
  +
  \left[\begin{array}{c|c}
      \Mat{K}^m & 0 \\
      \hline
      0 & 0 \\
  \end{array}\right]
\end{equation}

Where $\Mat{K}^m$ is the stiffness matrix of the triangular membrane.
The resulting
system of linear equations is then solved by direct LDL solver.

%}}}

%}}}


\section{Experiments} %{{{

To evaluate the model we have implemented in the
SOFA\footnote{www.sofa-framework.org} framework and performed a set of
numerical simulations. First local deformations were compared to the
results reported in literature to validate the method. Then the model of
the complete liver was subjected to natural global deformations to asses
the importance of the capsule on the deformation.

Material properties used in the simulations were for a porcine liver as
reported by Umale \cite{Umale2013}. The
parenchyma was modeled with elastic modulus 1980~Pa and the capsule has
elastic modulus 8~MPa and thickness 20~$\mu$m.
\TG{values for blood vessels}


\CD{If the experiments demonstrate that the stiffness of the material is different, the thickness is very very small... so we have to evaluate if it is really necessary to include the model of the capsule for both local and global deformations.}
\TG{I didn't undersand that. What do you mean?}

\subsection{Local Deformations} %{{{

During the contact with an instrument (probe, needle insertion, sharp
instruments, \ldots) a specific deformation occurs in the near vicinity of
the instrument. This type of deformation may not necessarily induce the
deformation of the object as a whole and therefore can be considered as
local. Correct material properties are not only important for amount of the
displacement but play also a crucial role in capturing the correct area of
the deformation or its profile near the instrument.
\TG{Does it make sense this way? I can add an image. :)}

One example of such local deformation occurs during the aspiration test
when the response of liver to the negative pressure is measured.
Hollenstein et al. \cite{Hollenstein2006} observed that the Glisson's
capsule has non-negligible influence on response of the tissue and modeling
only the parenchyma leads to overestimation of the deformation. We
reproduced the test with our model and acquired similar results.

The aspiration device consists of a tube with 1 cm in diameter and is able
to control internal pressure in the tube. The test is performed by
attaching the tube to the tissue and measuring the tissue response. We
have set up a simulation in the spirit of this test. 
The simulation is performed with a 15x15x15 mm cube of liver tissue and tube
with 1 cm in diameter. The pressure of 3 kPa is simulated inside the tube.

Figure \ref{fig-aspiration} shows the profiles of cuts in the middle of the
test cube. It can be seen that the
deformation for the model without capsule is much larger. This is in good
agreement with the data obtained by Hollenstein et al.

\Figure{4in}{aspiration}{TODO: aspiration test; fix figure, add screenshot}

%}}}




\subsection{Global Deformations}
\CD{ Experiments demonstrate that capsule plays a role at local level, but still, maybe not so important to model the capsule for more global deformation...
+ What are the applications for modeling the global deformations of the liver (training and planning simulation, physics-based registration, augmented reality,)}


\TG{!!!}

Test 1) Global deformation of liver under gravity
  * show that there is a difference between simple homogeneous material
    and material with capsule
  * include blood vessels

%}}}

\section{Conclusion} %{{{

A composite model for simulation of complete liver is presented in the
paper. The work of Peterlik et al. for vaskularized liver model was
extended by addig the capsule modeled with membrane FEM model mechanically
coupled with the parenchyma. To validate the model a simulation of local
deformations was performed and was found in an aggrement with the results
measured and published by Hollenstein et al.

While the influence of the capsule on local deformations was previously
studied in the literature, it's significance on the global scale
deformations of the liver remained unexplored. We have performed a set of
natural global deformations of the complete liver and shown that the
capsule, despite its small thickness, plays a significant role also in
global context.

%}}}

%
% ---- Bibliography ----
%

\bibliographystyle{splncs03}
\bibliography{bibdata}


\end{document}
% vim:set et sw=2 tw=75 fdm=marker fdl=3 fdc=4 isk+=_,-:
