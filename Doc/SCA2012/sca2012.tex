\documentclass{egpubl}
\usepackage{sca2012}

% --- for  Annual CONFERENCE
% \ConferenceSubmission % uncomment for Conference submission
% \ConferencePaper      % uncomment for (final) Conference Paper
% \STAR                 % uncomment for STAR contribution
% \Tutorial             % uncomment for Tutorial contribution
% \ShortPresentation    % uncomment for (final) Short Conference Presentation
%
% --- for  CGF Journal
% \JournalSubmission    % uncomment for submission to Computer Graphics Forum
% \JournalPaper         % uncomment for final version of Journal Paper
%
% --- for  EG Workshop Proceedings
\WsSubmission    % uncomment for submission to EG Workshop
% \WsPaper         % uncomment for final version of EG Workshop contribution
%
 \electronicVersion % can be used both for the printed and electronic version

% !! *please* don't change anything above
% !! unless you REALLY know what you are doing
% ------------------------------------------------------------------------

% for including postscript figures
% mind: package option 'draft' will replace PS figure by a filname within a frame
\ifpdf \usepackage[pdftex]{graphicx} \pdfcompresslevel=9
\else \usepackage[dvips]{graphicx} \fi

\PrintedOrElectronic

% prepare for electronic version of your document
\usepackage{t1enc,dfadobe}

\usepackage{egweblnk}
\usepackage{cite}

% For backwards compatibility to old LaTeX type font selection.
% Uncomment if your document adheres to LaTeX2e recommendations.
% \let\rm=\rmfamily    \let\sf=\sffamily    \let\tt=\ttfamily
% \let\it=\itshape     \let\sl=\slshape     \let\sc=\scshape
% \let\bf=\bfseries

% end of prologue

% ------------------------------------------------------------------------

\title[EG \LaTeX\ Author Guidelines]%
      {Fill in the Title} % TODO

% for anonymous conference submission please enter your SUBMISSION ID
% instead of the author's name (and leave the affiliation blank) !!
\author[...]
       {...} % TODO

% ------------------------------------------------------------------------

% if the Editors-in-Chief have given you the data, you may uncomment
% the following five lines and insert it here
%
% \volume{27}   % the volume in which the issue will be published;
% \issue{1}     % the issue number of the publication
% \pStartPage{1}      % set starting page


%-------------------------------------------------------------------------
\begin{document}

% \teaser{
%  \includegraphics[width=\linewidth]{eg_new}
%  \centering
%   \caption{New EG Logo}
% \label{fig:teaser}
% }

\maketitle

\begin{abstract} % TODO
   The ABSTRACT is to be in fully-justified italicized text, 
   between two horizontal lines,
   in one-column format, 
   below the author and affiliation information. 
   Use the word ``Abstract'' as the title, in 9-point Times, boldface type, 
   left-aligned to the text, initially capitalized. 
   The abstract is to be in 9-point, single-spaced type.
   The abstract may be up to 3 inches (7.62 cm) long. \\
   Leave one blank line after the abstract, 
   then add the subject categories according to the ACM Classification Index 
   (see http://www.acm.org/class/1998/).

% TODO
\begin{classification} % according to http://www.acm.org/class/1998/
\CCScat{Computer Graphics}{I.3.3}{Picture/Image Generation}{Line and curve generation}
\end{classification}

\end{abstract}





%-------------------------------------------------------------------------
\section{Introduction}

Long captions should be set as in Figure~\ref{fig:ex1}.

\begin{figure}[htb]
   % an empty figure just consisting of the caption lines
   \caption{\label{fig:ex1}
     'Empty' figure only to serve as an example of long caption requiring 
     more than one line. It is not typed centered but aligned on both sides.}
\end{figure}

%-------------------------------------------------------------------------
\subsection{References}

All graphics should be centered.

%%%
%%% Figure 1
%%%
%\begin{figure}[htb]
%  \centering
%  % the following command controls the width of the embedded PS file
%  % (relative to the width of the current column)
%  \includegraphics[width=.8\linewidth]{sampleFig}
%  % replacing the above command with the one below will explicitly set
%  % the bounding box of the PS figure to the rectangle (xl,yl),(xh,yh).
%  % It will also prevent LaTeX from reading the PS file to determine
%  % the bounding box (i.e., it will speed up the compilation process)
%  % \includegraphics[width=.95\linewidth, bb=39 696 126 756]{sampleFig}
%  %
%  \parbox[t]{.9\columnwidth}{\relax
%           For all figures please keep in mind that you \textbf{must not}
%           use images with transparent background! 
%           }
%  %
%  \caption{\label{fig:firstExample}
%           Here is a sample figure.}
%\end{figure}

%If your paper includes images, it is very important that they are of
%sufficient resolution to be faithfully reproduced.
%
%To determine the optimum size (width and height) of an image, measure
%the image's size as it appears in your document (in millimeters), and
%then multiply those two values by 12. The resulting values are the
%optimum $x$ and $y$ resolution, in pixels, of the image. Image quality
%will suffer if these guidelines are not followed.
%
%Example 1: 
%%
%An image measures 50\,mm by 75\,mm when placed in a document. This
%image should have a resolution of no less than 600 pixels by 900
%pixels in order to be reproduced faithfully.

%%------------------------------------------------------------------------
%\subsection{Copyright forms}
%
%You must include your signed Eurographics copyright release form
%when you submit your finished paper. We MUST have this form before
%your paper can be published in the proceedings.

%-------------------------------------------------------------------------

%\bibliographystyle{eg-alpha}
\bibliographystyle{eg-alpha-doi}

\bibliography{sca2012.bib}

\end{document}
