\documentclass{egpubl}
\usepackage{sca2012}

% --- for  Annual CONFERENCE
% \ConferenceSubmission % uncomment for Conference submission
% \ConferencePaper      % uncomment for (final) Conference Paper
% \STAR                 % uncomment for STAR contribution
% \Tutorial             % uncomment for Tutorial contribution
% \ShortPresentation    % uncomment for (final) Short Conference Presentation
%
% --- for  CGF Journal
% \JournalSubmission    % uncomment for submission to Computer Graphics Forum
% \JournalPaper         % uncomment for final version of Journal Paper
%
% --- for  EG Workshop Proceedings
\WsSubmission    % uncomment for submission to EG Workshop
% \WsPaper         % uncomment for final version of EG Workshop contribution
%
 \electronicVersion % can be used both for the printed and electronic version

% !! *please* don't change anything above
% !! unless you REALLY know what you are doing
% ------------------------------------------------------------------------

% for including postscript figures
% mind: package option 'draft' will replace PS figure by a filname within a frame
\ifpdf \usepackage[pdftex]{graphicx} \pdfcompresslevel=9
\else \usepackage[dvips]{graphicx} \fi

\PrintedOrElectronic

% prepare for electronic version of your document
\usepackage{t1enc,dfadobe}

\usepackage{egweblnk}
\usepackage{cite}

% For backwards compatibility to old LaTeX type font selection.
% Uncomment if your document adheres to LaTeX2e recommendations.
% \let\rm=\rmfamily    \let\sf=\sffamily    \let\tt=\ttfamily
% \let\it=\itshape     \let\sl=\slshape     \let\sc=\scshape
% \let\bf=\bfseries

% end of prologue

% ------------------------------------------------------------------------

\usepackage{ucs}
\usepackage[utf8x]{inputenc}

\usepackage{amsmath}

% Partial derivative
\newcommand{\deriv}[2]{\frac{\partial #1}{\partial #2}}

\newcommand{\Figure}[3]{%
\begin{figure}[htb]
  \centering
  \includegraphics[width=#1]{#2}
  \caption{\label{fig-#2}#3}
\end{figure}}


\title[Short title]%
      {Fill in the Title} % TODO

% for anonymous conference submission please enter your SUBMISSION ID
% instead of the author's name (and leave the affiliation blank) !!
\author[...]
       {...} % TODO

% ------------------------------------------------------------------------

% if the Editors-in-Chief have given you the data, you may uncomment
% the following five lines and insert it here
%
% \volume{27}   % the volume in which the issue will be published;
% \issue{1}     % the issue number of the publication
% \pStartPage{1}      % set starting page


%-------------------------------------------------------------------------
\begin{document}

% \teaser{
%  \includegraphics[width=\linewidth]{eg_new}
%  \centering
%   \caption{New EG Logo}
% \label{fig:teaser}
% }

\maketitle

\begin{abstract} % TODO
   The ABSTRACT is to be in fully-justified italicized text, 
   between two horizontal lines,
   in one-column format, 
   below the author and affiliation information. 
   Use the word ``Abstract'' as the title, in 9-point Times, boldface type, 
   left-aligned to the text, initially capitalized. 
   The abstract is to be in 9-point, single-spaced type.
   The abstract may be up to 3 inches (7.62 cm) long. \\
   Leave one blank line after the abstract, 
   then add the subject categories according to the ACM Classification Index 
   (see http://www.acm.org/class/1998/).

% TODO
\begin{classification} % according to http://www.acm.org/class/1998/
\CCScat{Computer Graphics}{I.3.3}{Picture/Image Generation}{Line and curve generation}
\end{classification}

\end{abstract}





%-------------------------------------------------------------------------
\section{Introduction}

Objects with thin structure form a very specific field of research. Because
they have high ratio between width and thickness methods from volumetric
deformation modeling cannot be immediately used. Such object occur in many
places e.g. textiles, paper or leaves. They are also very common in
anatomical structures for example in form of tubular objects like blood
vessels or colon, in the form of thin membranes, but also many others
(lense, skin, \ldots).

Methods based on continuum mechanics and namely theory of elasticity gained
on popularity ever since Terzopoulos \cite{Terzopoulos1987} presented his
work on elastic deformation modeling. While the methods using finite
elements and theory of elasticity are fairly popular in volumetric modeling
\footnote{TODO: do we need ref. here?} of deformations, they are still
frowned uppon in the area of thin structures. Instead either the
mass-spring models \cite{Volino2009} or bending models
\cite{Grinspun2003,Choi2007} are preffered. Unfortunately they have the
disadvantage that physical parameters of a material (like Young's modulus
and Poisson's ratio) cannot be used directly.

The usual argument agains FEM based methods being that they are too slow.
This is indeed true if lots of elements are used to discretize the object.
However, exploiting the bending property of the shell elements allows us to
achieve good results even with relatively small number of elements. We can
use high-polygonal mesh only for rendering or collision detection and
response. Similar idea has been already presented in the work of Bouthors
et al. \cite{Bouthors2007}. They used twinned meshes for animation of mesh
defined by implicit surface. Simple low-polygonal mesh was used for
mechanical FEM model and the mesh was subdivided in areas with high
curvature after every step of the simulation to produce finer mesh for
rendering. The underlaying FEM model is however not aware of the implicit
surface used. We build uppon the work of Comas \cite{Comas2010c} who used a
simple cubic polynomial to describe the element geometry and derived the
shell model from it.

The Comas's element has several drawbacks that we needed to address. Namely
it produces discontinuities between the elements on the edges, it is
missing one rotationonal degree of freedom and is not completly symetric in
context of deformations.

% TODO: cite Ubach & Oñate?

In this paper we present a new shell element based on the formulation of
Bézier triangle.

\subsection{Contribution}

% TODO
The main contributions of the paper are following:
\begin{itemize}
    % not really novel, similar model already in Ubach2010
    %\item shell element based on Bézier triangles
    \item Our element solves the problems of the element presented by Comas
        \cite{Comas2010c}.

    % TODO: needs better explanation
    % + two stage interpolation => Nodes (6DOFS) => bezier points => bezier
    %   triangle
    \item Two stage interpolation: 6DOF nodes -> bezier points -> bezier
        triangle

    % + separation between nodes and dofs
    \item The model separates nodes of the mesh from degrees of freedom of the
        control mesh. This allows mapping a surface on deformable skeleton.

    % + mapping: smooth surface results with less shells + collision response
    %   on smooth surface +  [? if I have time]  deformable on deformable:
    %   conjonctiva 2D motion on the surface of the shells
    \item Smooth maping of high resolution mesh on the curved surface of the
        shell. The mapped surfce can be used for collision detection and
        allows the propagation of proper collision response back onto the
        (low resolution) mechanical mesh.

\end{itemize}


\section{Preliminaries}

\subsection{Shell Element}

Shell element combines two types of deformations:

\begin{itemize}

    \item Elastic membrane: deformation in the plane of the element,
        described by in-plane displacements $u_x, u_y$ and rotation
        $\theta_z$ around the axis perpendicular to the plane of the
        element (Figure \ref{fig-membrane}).

    \item Bending plate: out of the plane deformation described by rotation
        around two in-plane axes $\theta_x, \theta_z$ and out of plane
        displacmenent $u_z$ (Figure \ref{fig-plate}).

\end{itemize}

\Figure{0.8\linewidth}{membrane}
{DOFs of elastic membrane element.}

\Figure{0.8\linewidth}{plate}
{DOFs of bending plate element.}

\subsection{Bézier Triangle} % {{{

In the rest of the document we define the surface over the triangle:
\begin{equation}
    (\xi_1,\xi_2) \in \Delta^2 = \left\{ (\xi_1,\xi_2)~|~\xi_1, \xi_2 \ge 0
        \mathrm{~and~} \xi_1+\xi_2 \le 1 \right\}
\end{equation}

The general $n$-th order Bézier triangle is defined as:

\begin{equation}
    T(\xi_1, \xi_2) = \sum_{0 \le i + j \le n} B^n_{i,j}(\xi_1,\xi_2) P_{i,j}
\end{equation}

where $P_{i,j}$ are the control points and $B^n_{i,j}$ are the bivairate
Bernstein basis functions defined as:

\begin{equation}
  B^n_{i,j} (\xi_1,\xi_2) =
    \binom{n}{i~j} \xi_1^i \xi_2^j (1-\xi_1-\xi_2)^{n-i-j},
\end{equation}
\begin{equation}
  \binom{n}{i~j} = \frac{n!}{i!j!(n-i-j)!}
\end{equation}

where $ 0 \le i+j \le n $.

Specificaly the cubic Bézier triangle ($n=3$) used in our model is
described by 10 control points and the surface is explicitly defined as (see
fig. \ref{fig-bezier}):

\begin{equation}\label{eq-cubicbez}
  \begin{split}
  T(\xi_1,\xi_2) & =
           + \xi_1^3\ P_1
           + \xi_2^3\ P_2
           + \xi_3^3\ P_3 \\
         & + 3\ \xi_1^2 \xi_2\ P_4
           + 3\ \xi_1^2 \xi_3\ P_5
           + 3\ \xi_2^2 \xi_3\ P_6 \\
         & + 3\ \xi_1 \xi_2^2\ P_7
           + 3\ \xi_1 \xi_3^2\ P_8
           + 3\ \xi_2 \xi_3^2\ P_9 \\
         & + 6\ \xi_1 \xi_2 \xi_3\ P_{10} \\
  \end{split}
\end{equation}

where $ \xi_3 = 1 - \xi_1 - \xi_2 $. By naming the respective values of
the Bernstein basis functions $n_i$ we can shortly express
\eqref{eq-cubicbez} as:

\begin{equation}\label{eq-cubicbez2}
    T = \sum_{i=1}^{10} n_i P_i
\end{equation}

\Figure{0.8\linewidth}{bezier}
{Cubic Bézier triangle with net of 10 control points.}

% }}}

\section{Bézier Shell Element} % {{{

\subsection{Control Points} % {{{

The Bézier triangles are defined by a net of control points, however the
simulated object is usually described by a triangular mesh. We construct
the Bézier mesh by taking into account the curvature of the triangular
mesh. When constructing the mesh a special care has to be taken to maintain
the continuity accross the nodes and edges. During the initialization phase
we partialy employ the method described in \cite{Ubach2010} to maintain
$C^0$ continuity on the edges. Each of the control points on the edge is
computed as the intersection of:

\begin{enumerate}
    \item The plane perpendicular to the normal at the vertex.
    \item The plane that contains the the curve of triangle's contour. The
        choise is arbitrary, but necessary to maintain $C^0$ continuity. 
        We choose the plane defined by the edge of the flat triangle and
        average of the two normals at vertices of the edge. 
    \item The plane perpendicular to the edge of the flat triangle placed at
        $1/3$ of the edge length.
\end{enumerate}

Notice that the points on the edge will be same for both triangles sharing
the edge.

We do not repeat this procedure during the simulation. For faster update of
the control points we rigidly attach the edge points to corner points. We
remember the position and orientation of the edge point relative to the
nearest corner point and use this information together with the rotation on
the corner points to reconstruct new position of control points. See Figure
\ref{fig-segments} for the correspondence between edge points and
associated corner points.

\Figure{0.8\linewidth}{segments}
{Correspondence between edge points and associated corner points shown in
blue.}

$C^1$ continuity can also be maintained if special care is taken when
computing the position of the central point (see \cite{Ubach2010}). For
simplicity we compute the central point followingly:

\begin{equation}\label{eq-central}
    P_{10} = \frac{1}{3}(\sum_{i=4}^9 P_i - \sum_{i=1}^3 P_i)
\end{equation}

% TODO: explain the equation eq-central? would require an image

That way the central point is slightly elevated and not in the plane of
other nodes thus keeping the curvature of the element.

%}}}

% TODO - nodes + DOFs (local + global position)

\subsection{Mechanics}

We base the displacement field $u$ on the formulation of the Bézier
triangle \eqref{eq-cubicbez}:

\begin{equation}\label{eq-bezU}
\begin{split}
  u & = U_1 ( \xi_1^3 + 3 \xi_1^2 \xi_2 + 3 \xi_1^2 \xi_3
        + 2 \xi_1 \xi_2 \xi_3 ) \\
    & + U_2 ( \xi_2^3 + 3 \xi_1 \xi_2^2 + 3 \xi_2^2 \xi_3
        + 2 \xi_1 \xi_2 \xi_3 ) \\
    & + U_3 ( \xi_3^3 + 3 \xi_1 \xi_3^2 + 3 \xi_2 \xi_3^3
        + 2 \xi_1 \xi_2 \xi_3 ) \\
    & + 3 \xi_1^2 \xi_2 (P_{4-1} \times \theta_1) \\
    & + 3 \xi_1^2 \xi_3 (P_{5-1} \times \theta_1) \\
    & + 3 \xi_1 \xi_2^2 (P_{7-2} \times \theta_2) \\
    & + 3 \xi_2^2 \xi_3 (P_{6-2} \times \theta_2) \\
    & + 3 \xi_1 \xi_3^2 (P_{8-3} \times \theta_3) \\
    & + 3 \xi_2 \xi_3^2 (P_{9-3} \times \theta_3) \\
    & + 2 \xi_1 \xi_2 \xi_3 (P_{10-1} \times \theta_1 +
        P_{10-2} \times \theta_2 +
        P_{10-3} \times \theta_3)
\end{split}
\end{equation}

where $U_1, U_2, U_3$ are the translational displacment vectors of the
corner nodes of the element, $\theta_1, \theta_2, \theta_3$ are the angular
displacements of the corner nodes and $P_{i-j}$ are the segments from
corner nodes to other nodes of the control mesh:

\begin{equation}
    P_{i-j} = P_j-P_i, \quad i \in [4;10], j \in [1;3]
\end{equation}

Using the plate theory we can now compute the strain-displacement matrices
from the displacement field. From the Cauchy's strain tensor for the
elastic membrane:

\begin{equation}
    J_m = \left[ \begin{matrix}
        \deriv{u_x}{x} \\
        \deriv{u_y}{y} \\
        \deriv{u_x}{y} + \deriv{u_y}{x}
    \end{matrix} \right]
\end{equation}

and from Kirchhoff-Love theory for thin plates:

\begin{equation}
    J_b = \left[ \begin{matrix}
        - z \deriv{^2 u_z}{x^2} \\
        - z \deriv{^2 u_z}{y^2} \\
        - 2z \deriv{^2 u_z}{xy}
    \end{matrix} \right]
\end{equation}

Assuming constant thickness $t$ of the element and integrating over the volume
of the element we compute the stiffness matrices for the elastic membrane
and bending plate respectively:

\begin{align}
    \label{eq-Km}
    K_m & = \iiint_V J_m^T M J_m \, \mathrm{d} V \\
    \label{eq-Kb}
    K_b & = \iiint_V J_b^T M J_b \, \mathrm{d} V
\end{align}

where $M$ is the material matrix. We use Hooke's law.

Because the deformation field for the shell is non-linear \eqref{eq-bezU}
the integrals \eqref{eq-Km} and \eqref{eq-Kb} have to be computed using
numerical integration. In our implementation we employ 4-point Gaussian
quadrature for integration over triangle area.
% TODO: add ref
% TODO: add abscissas/weights?
%  \xi_1        \xi_2           w
% 0.211324865, 0.166666667 0.197168783
% 0.211324865, 0.622008467 0.197168783
% 0.788675134, 0.044658198 0.052831216
% 0.788675134, 0.166666667 0.052831216

To keep the system linear during simulation we compute the forces in
corotational frame.

% }}}

\section{Mechanical Mapping} % {{{

% TODO: mention it is 6 DOF <-> 3 DOF

Because of the bending property of the shells relatively few elements are
necessary to simulate curved surface. To enrich the visual experience from
the simulated object it is desirable to use more triangles in the areas
with high curvature during rendering. To do that we can map a high
resolution mesh onto the mechanical mesh. For every vertex of the high
resolution mesh we first find the triangle on the mechanical mesh that is
closest to the vertex and then assign barycentric coordinates on the
triangle to this vertex. After every step of the simulation the high
resolution mesh is updated using the assigned barycentric coordinates and
the function of the surface \eqref{eq-cubicbez}.

mapping velocities (necessary for collisions?)


Differentiating \eqref{eq-cubicbez2} by time we get:

\begin{equation}
    \dot{T} = \sum_{i=1}^{10} n_i \dot{P_i} = \sum_{i=1}^{10} n_i V_i
\end{equation}

Which means we use the same expression to interpolate the velocities only
by substituting the velocities at control points. For the corner nodes we
already have the velocities $V_i$. For the internal nodes we need to compute
them. In case of the edge nodes it is a problem of computing the velocity
of a point attached to the moving rigid body:

\begin{eqnarray}
    V_i = V_j + \omega_j \times (P_i - P_j)
\end{eqnarray}

where $j$ is the index of the corner node this edge node is attached to and
$\omega_j$ is the angular velocity at the node. For the central node we
again use the interpolation as in \eqref{eq-central}.

* reverse mapping forces for collision response

We compute the influence of force $F$ on the corners through all the
control points by the formula:

\begin{equation}
    F_j += F n_i
\end{equation}

and the torques applied through edge control points by:

\begin{eqnarray}
    \theta_j += (P_i - P_j) \times F
\end{eqnarray}

where $j$ is the index of the corner node associated with control point
$P_i$.

And finaly we add the torques through the central control point with:

\begin{eqnarray}
    \theta_1 += ((P_4 - P_1) + (P_5 - P_1)) \times F \\
    \theta_2 += ((P_6 - P_2) + (P_7 - P_2)) \times F \\
    \theta_3 += ((P_8 - P_3) + (P_9 - P_3)) \times F
\end{eqnarray}

% }}}


\section{Results}

proof of symmetry

some other examples?

computation speed


\section{Discussion}


% = = = = = = = = = = = = = = = = = = = = = = = = = = = = = = = = = = = = =
%
% Benefits:
% + C^0 continuity on the edges
% + Perfectly symmetric results for the element (it is not the case with
%   "classical shells")
% + Better conditioning (comparison between Comas model and ours in term of
%   convergence of CG...) * Good trade-off between continuity of interpolation
%   and spasticity: only 18 Dofs per elements for an interpolation with 3d
%   order
% + polynomials [Bezier triangle has theoretically 10 points x 3 dofs=> 30 dofs]
%
% Limitations:
% + Continuity C1 at the nodes but C0 on the edges
% + The segments are at fixed distance from corner nodes, may lead to
%   uneven curvature for too stretched or shrinked elements 
%
% Applications:
% + cataract surgery
% + arterial vessel (MICCAI)
% + vagina/uterus?
%
% = = = = = = = = = = = = = = = = = = = = = = = = = = = = = = = = = = = = = = =



%Long captions should be set as in Figure~\ref{fig:ex1}.
%
%\begin{figure}[htb]
%   % an empty figure just consisting of the caption lines
%   \caption{\label{fig:ex1}
%     'Empty' figure only to serve as an example of long caption requiring 
%     more than one line. It is not typed centered but aligned on both sides.}
%\end{figure}
%
%All graphics should be centered.
%
%%%
%%% Figure 1
%%%
%\begin{figure}[htb]
%  \centering
%  % the following command controls the width of the embedded PS file
%  % (relative to the width of the current column)
%  \includegraphics[width=.8\linewidth]{sampleFig}
%  % replacing the above command with the one below will explicitly set
%  % the bounding box of the PS figure to the rectangle (xl,yl),(xh,yh).
%  % It will also prevent LaTeX from reading the PS file to determine
%  % the bounding box (i.e., it will speed up the compilation process)
%  % \includegraphics[width=.95\linewidth, bb=39 696 126 756]{sampleFig}
%  %
%  \parbox[t]{.9\columnwidth}{\relax
%           For all figures please keep in mind that you \textbf{must not}
%           use images with transparent background! 
%           }
%  %
%  \caption{\label{fig:firstExample}
%           Here is a sample figure.}
%\end{figure}

%If your paper includes images, it is very important that they are of
%sufficient resolution to be faithfully reproduced.
%
%To determine the optimum size (width and height) of an image, measure
%the image's size as it appears in your document (in millimeters), and
%then multiply those two values by 12. The resulting values are the
%optimum $x$ and $y$ resolution, in pixels, of the image. Image quality
%will suffer if these guidelines are not followed.
%
%Example 1: 
%%
%An image measures 50\,mm by 75\,mm when placed in a document. This
%image should have a resolution of no less than 600 pixels by 900
%pixels in order to be reproduced faithfully.

%%------------------------------------------------------------------------
%\subsection{Copyright forms}
%
%You must include your signed Eurographics copyright release form
%when you submit your finished paper. We MUST have this form before
%your paper can be published in the proceedings.

%-------------------------------------------------------------------------

%-------------------------------------------------------------------------

%\bibliographystyle{eg-alpha}
\bibliographystyle{eg-alpha-doi}

\bibliography{sca2012.bib}

\end{document}
