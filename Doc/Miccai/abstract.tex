% Pediatric cardiac surgery of congenital heart defects is particularly challenging since final decisions on the surgical procedure take place just in time during open-heart operations.
% The work presented in this paper aims at proposing an interactive simulation software dedicated to preoperative planning.
% This new tool would permit to evaluate several possible surgical procedures on virtual anatomy of the patient and anticipate the decisions.
% The surgery involves a reconfiguration of the great arteries.
% During the surgery, the vessel walls are deformed, cut, and reconnected to other vessels or closed with an artificial patch.
% We propose to anticipate the geometry changes of the vessel walls using a simulation based on thin-shell elements.
% We demonstrate that our approach allows for simulating the results of low-level surgical procedures, with a unified strategy for joining the vessel models and simulate their subsequent behavior.

Congenital heart defect corrective surgeries are particularly challenging since final decisions on concrete surgical procedures to repair malformations take place just in time during open-heart surgeries. In this paper we propose an interactive simulation system for preoperative planning. The system allows to evaluate outcomes of different surgical procedures in advance to the actual operation. During surgery, blood vessel walls are deformed, cut, reconnected to other blood vessels and patches are attached to repair malformations. We use an approach for simulating the results of these geometric changes based on thin shell elements. We show that our method enables predictive simulation of any low-level surgical procedures by using a novel approach for joining of blood vessels without springs.