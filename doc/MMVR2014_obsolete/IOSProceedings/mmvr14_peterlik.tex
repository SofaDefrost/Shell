\documentclass{IOS-Book-Article}

\usepackage{mathptmx}

%\usepackage{times}
%\normalfont
%\usepackage[T1]{fontenc}
%\usepackage[mtplusscr,mtbold]{mathtime}
%

%\usepackage{amsmath}
%\usepackage{amssymb}
\usepackage{graphicx}
%\usepackage{subfigure}
%\usepackage{rotating}
%\usepackage{array}
%\usepackage[lined,linesnumbered]{algorithm2e}%
%\usepackage{algorithmic}%
%\usepackage{multirow}
%\usepackage{amsmath}
%\usepackage[utf8x]{inputenc}
%\usepackage[T1]{fontenc} % pour les caracteres francais dans les remarques
%\usepackage{textcomp}
%\usepackage[usenames]{color}
%
\newcommand{\Vec}[1]{\mathbf{#1}}
\newcommand{\Mat}[1]{\mathbf{#1}}


\begin{document}
\begin{frontmatter}              % The preamble begins here.

%\pretitle{Pretitle}
\title{Complete Real-Time Liver Model Including Glisson's Capsule,\\
 Vascularization and Parenchyma}
\runningtitle{Complete Real-Time Liver Model}
%\subtitle{Subtitle}

\author[A]{\fnms{Igor} \snm{Peterl\'ik}}%
%\thanks{Corresponding Author: Book Production Manager, IOS Press, Nieuwe Hemweg 6B,
%1013 BG Amsterdam, The Netherlands; E-mail:
%bookproduction@iospress.nl.}},
\author[B]{\fnms{Tom\'a\v{s}} \snm{Golembiovsk\'y}}
\author[C]{\fnms{Christian} \snm{Duriez}}
and
\author[C]{\fnms{St\'{e}phane} \snm{Cotin}}

\runningauthor{I. Peterlik et al.}
\address[A]{Institut Hospitalo-Universitaire, Strasbourg, France}
\address[B]{Masaryk University, Czech Republic}
\address[C]{SHACRA Team, Inria, France}

\begin{abstract}
Accurate biomechanical modeling of liver is of a paramount interest for pre-operative planning or computer-aided per-operative guidance.
However, such simulations remain a challenging task, since the organ is composed of three constituents: parenchyma, vascularization and Glisson's capsule,
each having different mechanical properties.

In this paper we propose a complete liver model, where the parenchyma is modeled as a finite element volume, the vessels are modeled as series of beam elements,
and the corotational membrane elements based on constant-strain formulation are used for capsule. These components are coupled mechanically as shown in the paper.
We show that using this model we are able to reproduce accurately biomechanical tests, while maintaining the real-time aspect of the simulation.
\end{abstract}

\begin{keyword}
Simulation , modeling, liver, biomechanics
\end{keyword}
\end{frontmatter}

\thispagestyle{empty}
\pagestyle{empty}

\section*{Introduction}%
According to the statistics, nearly 100,000 European citizens die of cirrhosis or liver cancer each year. 
Although new methods 
%such as radio-frequency- and cryo-ablation known 
in the interventional radiology 
seem to be promising, surgery remains the option that offers the foremost success rate against these pathologies. 
Nevertheless, surgery is not always performed due to several limitations, in particular the determination 
of accurate eligibility criteria for the patient. 
%In this context, the pre-operational planning becomes a crucial task having a significant impact on the treatment. 

Computer-aided physics-based medical simulation has proven to be an extremely useful technique in the area of medical training. 
Whereas generic models are usually required in training simulators, accurate patient-specific modeling
becomes necessary as soon as computer simulation is to be employed in the pre-operative planning. 
%At the same time, 
%interactivity of such models remains an important aspect, requiring real-time simulation which is often difficult to 
%achieve given the complexity of soft tissues. 
When simulating the behavior of human liver, the task of real-time accurate modeling is challenging, mainly because of the complex structure 
of this organ composed of three main constituents: \emph{parenchyma}, \emph{vascular networks} and \emph{Glisson's capsule}.
The parenchyma has certainly been the most studied component of the liver; actually researchers agree on hyperelastic 
properties of the tissue, for which the mechanical parameters have been reported for example in~\cite{Kerdok2006}.%,Gao2009}. 
Moreover, several methods have been proposed to model the hyperelastic behaviour at real-time rates, such as multiplicative Jacobian decomposition
introduced in~\cite{Marchesseau2010}.
%
%\TG{The following paragraph is optional. Unnecessary but adds to the global context}
The mechanical importance of the vascular structures in liver is studied in~\cite{Peterlik2012}. It shows that the 
influence of vessels on the mechanical behaviour of the organ is significant. %, mainly if large deformations occur. 
%Since a detailed modeling of the vessels would be extremely costly (mainly because of the small thickness of the vessel wall, 
%the authors propose a composite model allowing for real-time simulation of entire liver with vascularization.
Since a detailed modeling of the vessels would be extremely costly (mainly because of the small thickness of the vessel wall), 
the authors propose a composite model allowing for real-time simulation.

Rather a small number of studies have been conducted dealing with the third liver constituent, the  Glisson's capsule.
Quantitative results of experiments on a porcine liver have been published in~\cite{Umale2011}; the measurements indicate that although being very 
thin (10--20$\mu m$), the capsule shows to be stiff in tensile tests: the Young's modulus of the capsule reported to be $8.22\pm3.42$\,MPa 
exceeds the values for the parenchyma by three orders of magnitude.
This suggests that the mechanical influence of the membrane on the liver behavior is not negligible.
%
In~\cite{Hollenstein2006}, a local influence of the capsule has been measured using a special aspiration device. The study was then repeated 
in vivo on human patients during the operation, confirming the mechanical importance of the membrane~\cite{Ahn2010,Nava2008}.
%To our best knowledge, no attempt has been made to demonstrate the role of the Glisson's capsule on the global behaviour of 
%the liver, mainly if the organ undergoes large deformations which is often the case during the surgery.

In this paper we present preliminary results of our work on the complete liver model. 
We present a real-time composite model accounting for parenchyma and Glisson's capsule, compatible with previously proposed vascularized model~\cite{Peterlik2012}.
We also show that the capsule has a significant impact on the mechanical response of the organ. 
%The model is based on two different finite element representations for each constituent coupled together.
%The main contribution of the paper is twofold: first, we present a real-time model of the liver, including
%the parenchyma, vascularization and Glisson's capsule. The model is based on three different finite element representations for each constituent,
%linked together via mechanical coupling.
%We show that the model mimics the \emph{local} experiments described in~\cite{Hollenstein2006}.
 
%Second, we use the complete model of the liver to demonstrate the \emph{global} influence of the Glisson's
%capsule via simulation: using a specific model of a porcine liver built from CT contrast-enhanced data, we show that there is a significant 
%difference in the response of the model with and without the capsule in the case when the liver undergoes large deformations. 

%The paper is organized as follows: first we describe the proposed model of liver with capsule. 
%Second, we validate our model in context of local deformations by reproducing the aspiration test described in~\cite{Hollenstein2006}. We conclude by summarizing future steps towards the complete liver model.
%Finally, we demonstrate that in spite of its small thickness, the Glisson's 
%capsule has a global influence on the liver undergoing large deformations. 
%
%
\section{Methods \& Materials}
\label{sec:methodology}
\vspace{-6pt}

% In this section we describe the construction of a composite model based on two components: 
% tetrahedral FE model of the  parenchyma and triangular membrane elements used for the capsule.

% \subsection{Biomechanical Models and Coupling}
It is known that the parenchyma exhibits non-linear viscoelastic behavior \cite{Marchesseau2010}.
However, as we are mainly interested in the static equilibrium, we do not model the time-dependent
phenomena related to viscosity. In the simulation we employ corotational finite elements~\cite{Felippa2005}.
Although it relies on linear stress-strain relationship, large displacements including rotations are modeled correctly.
% Briefly, the motion of each element $e$ is decomposed into rigid rotation $\Mat{R}^e$ and local deformation in each step of the simulation. 
% The rotations are then used to update each local element stiffness matrix as $\Mat{R}^e\Mat{K}_0{\Mat{R}^e}^{\top}$
% whereas the deformations are used to compute the linear strain in the local corotational frame.
% There are several ways of computing the corotational frame for elements; we rely on
% the geometrical method proposed in \cite{Nesme2005}.

% \subsection{Parenchyma} %{{{

% It is known that the parenchyma exhibits non-linear viscoelastic behaviour \cite{Marchesseau2010}.
% However, as we are mainly interested in the static equilibrium, we do not model the time-dependent
% phenomena related to viscosity.
%
% % %However, we employ simpler corotational model as we are not so much
% % %interested in time-dependent behaviour but rather in static equilibrium
% % %under certain conditions.
% % %We also rely on the vascularized model of the parenchyma proposed by
% % %Peterl\'{i}k et al. \cite{Peterlik2012}.
%
% The parenchyma is modeled using corotational finite elements~\cite{Felippa2005}.
% Although it relies on linear stress-strain relationship, large displacements including rotations are modeled correctly. 
% While in the full non-linear formulation the stiffness matrix relates the forces $\Vec{f}$ and 
% displacements $\Vec{u}$ as $\Vec{f} = K(\Vec{u})$, the corotational model 
% requires the stiffness matrix $\Mat{K}_0$ of the system to be computed only once before the simulation begins. 
% Then, in each step, the motion of each element $e$ is decomposed into rigid rotation $\Mat{R}^e$ and local deformation. 
% The rotations are then used to update each local element stiffness matrix as $\Mat{R}^e\Mat{K}_0{\Mat{R}^e}^{\top}$
% whereas the deformations are used to compute the linear strain in the local corotational frame.
% There are several ways of computing the corotational frame for elements; we rely on
% the geometrical method proposed in \cite{Nesme2005}.
% % % NOTE: This description of corotational method is very simplified and could be extended.
% % 
% % %The model of the vascularization is based on linear beams 
% % %with local frames of reference~\cite{Duriez2006}; in many aspects it's similar to the
% % %corotational formulation described above. As such the model also handles geometric
% % %non-linearities in the deformation. Through a specification of cross section and moments of inertia, 
% % %the model can account for the specific properties of the blood vessels. 
% % 
% % %The beam-based model of vessels is mechanically coupled to the parenchyma as described in~\cite{Peterlik2012}. 
% % %The coupling assumes that there is no relative motion between the vessels and the surrounding parenchyma. 
% % %During the simulation, the nodes of linked beams are first displaced and rotated according to the actual motion of associated tetrahedra. 
% % %As the deformation of beams results in mechanical response represented by forces and torques, these are propagated back to 
% % %the tetrahedral FE model. 
% % 
% % %}}}

% \subsection{Glisson's Capsule} %{{{
% \label{ss:capsuleModel}
The thickness of the Glisson's capsule is relatively small: the values in range of 10-20
$\mu$m have been reported in~\cite{Umale2011}.
It is not possible to model such thin structure with classical tetrahedral
elements:
%if the real-time aspect of the simulation is to be preserved.
%Furthermore, modeling both the tissue and the capsule 
the model would require an extremely 
dense mesh to avoid numerical instabilities and thus would significantly
violate the speed requirements imposed for medical simulators.
Instead, modeling the capsule with two-dimensional elements that abstract from the
thickness in the third dimension seems
to be a natural choice. In the elasticity theory, this functionality is usually provided by membrane and shell elements.
Based on the observation of its behaviour, we also
assume negligible bending forces and propose a model based on membrane
elements. 
To maintain simplicity of the composite model we choose simple triangular
elements with constant strain (CST).

The computation of elastic stiffness matrix follows the common derivation
%
\begin{eqnarray}
  \Mat{K}^m & = & \int_V \Mat{B}^T \Mat{E} \Mat{B} dV     \label{mem1} \\
            & = & h \int_A \Mat{B}^T \Mat{E} \Mat{B} dA   \label{mem2} \\
            & = & h A \Mat{B}^T \Mat{E} \Mat{B}           \label{mem3}
\end{eqnarray}
%
where $\Mat{B}$ is the strain-displacement matrix, $\Mat{E}$ the material
matrix, $h$ is the thickness and $A$ area of the element. In the previous
\eqref{mem2} follows from the fact that we assume constant thickness of the
element and \eqref{mem3} follows from the fact that the strain-displacement
matrix is constant in our case. The strain-displacement matrix for the CST
element can be expressed as:
%
\begin{equation}
  \Mat{B} = \frac{1}{2A} \begin{bmatrix}
    y_{23} & 0      & y_{31} & 0      & y_{12} & 0 \\
         0 & x_{32} & 0      & x_{13} & 0      & x_{21} \\
    x_{32} & y_{23} & x_{13} & y_{31} & x_{21} & y_{12}
  \end{bmatrix}
\end{equation}
%
The values $x_{ij} = x_i - x_j$ and $y_{ij} = y_i - y_j$ are computed from
the $x$ or $y$ coordinates of the nodes $i,j$ of the triangular element.
The reader can refer to the respective literature~\cite{Felippa2003} for more thorough
description.

Similarly as with model of parenchyma we use linear elastic material and employ
the corotational formulation for the CST elements.

%}}}

%\subsection{Mechanical Coupling between Parenchyma and Capsule} %{{{
The literature reports high cohesion between capsule and parenchyma.
Based on this property we assume there is no relative motion of the capsule \wrt\ the parenchyma.
Although an arbitrary surface mesh could be used to model the capsule, we exploit 
the fact that the parenchyma is modeled by tetrahedral elements having
triangular faces. Thus, as the boundary of the volumetric mesh is already
triangulated, we simply employ the triangles on the mesh surface to model the capsule.

Using directly the boundary of the tetrahedral mesh does not only solve the
problem of building the surface mesh, but has one more advantage: the nodes
of the triangular mesh coincide with the nodes of the tetrahedral mesh, so no projection of one mesh onto the other is needed.
and the stiffness matrices for capsule and parenchyma are then easily assembled together by adding the mechanical contribution 
of a triangle to the corresponding tetrahedron.
% and solved as one system.
%
%\CD{The following may appear as "trivial"... maybe we can remove this part at the end if we need space} 
Without the loss of generality we can assume the tetrahedron consists of
nodes $p_1, p_2, p_3$ and $p_4$ and the boundary triangle has nodes $p_1, p_2$
and $p_3$. We can reorder the degrees of freedom so that the stiffness
matrix $\Mat{K}^t$ for the tetrahedron can be written as:
%
\begin{equation}
  \Mat{K}^t = \left[\begin{array}{c|c}
      \Mat{K}^t_{1-3,1-3} & \Mat{K}^t_{1-3,4} \\
      \hline
      \Mat{K}^t_{4,1-3} & \Mat{K}^t_{4,4} \\.
  \end{array}\right]
\end{equation}
%
Hence, the assembled stiffness matrix for the element is
%
\begin{equation}
  \Mat{K} = \left[\begin{array}{c|c}
      \Mat{K}^t_{1-3,1-3} & \Mat{K}^t_{1-3,4} \\
      \hline
      \Mat{K}^t_{4,1-3} & \Mat{K}^t_{4,4} \\
  \end{array}\right]
  +
  \left[\begin{array}{c|c}
      \Mat{K}^m & 0 \\
      \hline
      0 & 0 \\
  \end{array}\right]
\end{equation}
%
where $\Mat{K}^m$ is the stiffness matrix of the triangular membrane.

The resulting system of linear equations is solved by direct solver based on Cholesky decomposition.

%Alternatively, one can use a mechanical coupling similar to the one used in
%\cite{Peterlik2012} to be able to use an arbitrary surface mesh. Nevertheless,
%for conforming triangular mesh both methods lead to the same solution.

%}}}

%}}}

% \subsection{Simulation} %{{{
% The model has been implemented SOFA\footnote{www.sofa-framework.org} and a set of
% numerical simulations was performed.
%In this section we provide comparison of local deformations with the
%results reported in literature to validate the method.
%Second, to show that in spite of its very small thickness the membrane cannot be
%neglected even in the context of global deformations and its overall
%stiffness plays an important role, the model of
%the complete liver was subjected to global deformations.

% During the contact with an instrument such as probe, needle, scalpel and others,
% specific deformations take place in the vicinity of
% the instrument. This type of deformation may not necessarily induce the
% deformation of the object as a whole and therefore can be considered as
% local. Correct material properties are not only important to quantify the
% displacement, but also play an important role in capturing the correct area of
% the deformation or its profile near the instrument.

% Since the tube is in direct contact with the tissue, uni-lateral constraints with friction were chosen 
% to model this interaction properly. We opted for a method based on \emph{non-linear complementarity problem}  (NLCP)
% where the non-linearity is introduced due to the friction. The NLCP
% allows for solution of the Signorini's problem to avoid any interpenetration between the colliding 
% objects (see~\cite{Duriez2006b} for details). Since NLCP requires explicitly the computation of compliance matrix which 
% is homogeneous to the inverse of the stiffness matrix, we employed a direct solver based on LDL decomposition to 
% solve the system and compute the inverse matrices. %
%
%\addtolength{\textheight}{-1cm}
In order to show an improvement on existing predictive simulation systems for CHD corrective surgeries, we want to demonstrate that our blood vessel model based on thin shell elements
\begin{enumerate}
\item physically meets the requirements for predictive simulation results,
\item converges to these results with a low number of elements,
\item is able to universally simulate low-level surgical procedures.
\end{enumerate}
We close this section with a discussion about limitations and constraints of our simulation approach in terms of generation and dynamic remeshing of thin shell element meshes.

A quantitative validation by means of a comparision to real pre- and post-surgery image data is impossible at present. This is mainly due to ethical issues that arise during image data acquirement of infants, i.e. necessary heart rate lowering medications or exposure to radiation. There is no such data available that is obtained shortly after surgeries in infants where the growth of the patient would not interfere comparison. However, a qualitative comparison to real image data acquired months after a surgical intervention, which confirms a principal suitability of predictive simulations for surgery planning, can be found in \cite{Li2009}.

\subsection{Predictive Simulation Results}

\begin{figure}[tbh]
    \centering
    \begin{tabular}{cc}
     \includegraphics[width=0.3\columnwidth]{img/compare-bend.png}
      &
      \includegraphics[width=0.3\columnwidth]{img/compare-twist.png}
      \\
      \includegraphics[width=0.3\columnwidth]{img/compare-bend-other.png}
      &
      \includegraphics[width=0.3\columnwidth]{img/compare-twist-other.png}
    \end{tabular}
    \caption{Simulation of bending and twisting of an elestic tube. Top row: Thin shell element model; Bottom row: Real elastic tube manipulated by hand and Cosserat rod-based hybrid model \cite{Li2009} (Credits for the bottom images: Li et al.).}
    \label{fig-deformations}
\end{figure}

Figure \ref{fig-deformations} shows bending and twisting of an elastic tube manipulated by hand compared to the simulation results of \cite{Li2009} and our method. These kinds of blood vessel deformations are likely to occur during surgery. Our results are close to the real deformations as expected due to the physically based formulation of thin shell elements.

\subsection{Convergence}

\begin{figure}[tbh]
  \centering
  \begin{tabular}{cccc}
    \includegraphics[width=0.24\columnwidth]{img/twist-06-cg.png}
    &
    \includegraphics[width=0.24\columnwidth]{img/twist-08-cg.png}
    &
    \includegraphics[width=0.24\columnwidth]{img/twist-16-cg.png}
    &
    \includegraphics[width=0.24\columnwidth]{img/twist-31-cg.png}
    \\
    \includegraphics[width=0.24\columnwidth]{img/twist-06w-cg.png}
    &
    \includegraphics[width=0.24\columnwidth]{img/twist-08w-cg.png}
    &
    \includegraphics[width=0.24\columnwidth]{img/twist-16w-cg.png}
    &
    \includegraphics[width=0.24\columnwidth]{img/twist-31w-cg.png}
  \end{tabular}
  \caption{Convergence of a twisted, elastic tube with 6, 8, 16 and 31 vertices
  along its circumference (120, 208, 832 and 3038 thin shell elements). Top row: High polygon count mesh mapped to simulation results of coarser shell element meshes (bottom row).}
  \label{fig-convergence}
\end{figure}

It is important to minimize the number of shell elements to reduce computation time. This is the key factor to enable an interactive simulation system that is accepted by surgeons to simulate multiple surgical approaches in succession. The bending capabilities of thin shell elements allow to reduce their number and still representing the tubular structure of blood vessels while maintaining their physical behavior appropriately. Figure \ref{fig-convergence} shows simulation results of a strongly deformed tube represented by different numbers of thin shell elements. With only eight vertices along the circumference of the elastic tube the simulation result is comparable to a simulation result with high element count.

\subsection{Low-Level Surgical Procedures}

\begin{figure}[tbh]
  \centering
  \includegraphics[width=\columnwidth]{img/surgery.png}
  \caption{Simulation of different surgical procedures for coarctation repair of an aortic arch. Left image: Overview of the simulation scene consisting of an aortic arch (AO), a left subclavian artery (LSA) and a pulmonary artery (PA). The coarctation can be seen next to the LSA and the dashed line indicates where the aorta is to be incised for a subclavian flap aortoplasty; The vertical image pairs show the scene before and after different surgical interventions. From left to right: Subclavian flap aortoplasty -- the subclavian flap is used as an organic onlay patch over the area of coarctation; Patch aortoplasty -- the incision is opened by attached springs only for visualisation purposes. The patch is sutured in place as shown; End-to-end anastomosis -- the coarctation is resected and the loose ends of the AO sutured together. Due to deformation of the AO there is a collision with the PA, causing slight deformations on both blood vessels. }
  \label{fig-surgery}
\end{figure}

We manually modeled an aortic arch based on real image data to be able to demonstrate suitability of our joining approach for different kinds of low-level surgical procedures on a concrete example. Therefore we chose a coarctation of an aortic arch that can be repaired by completely different surgical approaches that heavily alter the surface of the blood vessel \cite{Dodge2000}. Results can be examined in figure \ref{fig-surgery}. We show that our topological method for joining of shell elements through a controlled relaxation to their rest shape is able to handle different kinds of connections universally and in a correct manner.

\subsection{Limitations and Constraints}

Finite element methods including thin shell elements have some constraints in common regarding the simulation mesh quality. A homogeneous neighborhood of equally sized elements is desirable. Especially when minimizing the number of elements, high mesh quality is necessary to maintain convergence and stability of the simulation. We showed that an elastic tubular structure can be simulated appropriately for CHD corrective surgeries with only a few elements. To allow for arbitrary incisions under such circumstances, the mesh must be remeshed in a way that its elements align along an incision. Mesh quality must be retained during remeshing and therfore the whole mesh may be affected. A limitation of our joining approach is that we need the same number of elements along connecting edges. Appropriate remeshing strategies must be developed to take full benefit of our proposed simulation approach. However, as a fallback, element count could be increased which will have negative impact on the simulation speed.

%Example with 3 diff. surgical procedures
%Parameters
%Limitations (getting image data, meshes, dynamic remeshing, same number of nodes)
%
%
We introduced a new approach for predictive simulation of CHD corrective surgeries. It is based on thin shell elements and a relaxation method for joining of blood vessels and patches by topologically connecting the vertices of joining edges of the simulation meshes. All necessary low-level surgical procedures can be performed using this method. We showed that we need only a small number of thin shell elemenets to achieve physically accurat simulation results which enable fast simulations. Special care has to be taken during initial mesh generation and dynamic remeshing when cutting blood vessels since the size of newly generated thin shell elements should roughly equal the size of their neighbor elements which is a common constraint for finite element simulations. To truly benefit from our simulation approach, future work has to be done in terms of remeshing, to keep a high quality of the mesh during the simulation.%

%
\bibliographystyle{plain}
\bibliography{../bibdata}%

%\begin{thebibliography}{99}
%
%\bibitem{r1}
%\textit{Scientific Style and Format: The CBE manual for authors,
%editors and publishers}. Style Manual Committee, Council of Biology Editors.
%Sixth ed. Cambridge University Press, 1994.
%
%\bibitem{r2}
%L.U. Ante, Cem surgere: Surgite postquam sederitis, qui manducatis panem doloris,
%\textit{Omnes} \textbf{13} (1916), 114--119.
%
%\bibitem{r3}
%T.X. Confortavit, \textit{Seras}, Portarum, New York, 1995.
%
%\bibitem{r4}
%P.A. Deus, Ater hoc et filius et mater praestet nobis,
%\textit{Paterhoc} \textbf{66} (1993), 856--890.
%
%\end{thebibliography}
\end{document}
