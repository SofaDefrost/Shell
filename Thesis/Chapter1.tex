\chapter{Introduction to medical simulation}
\label{chap1}


\section{General context}

Virtually everyone knows this situation where a close relative has to go to the hospital, either for a simple routine check or a serious surgery. Hospitals are meant to be reinsuring, a refuge where people's health improves, where people get fixed. Obviously, patient safety is a critical concern in the medical industry. Yet, patients have been known to suffer injuries or even death due to errors of judgement or a lack of care and training. As an example, a report published in 2000 by the Institute of Medicine in the United States \citep{Kohn00} described two studies carried out in the 1990s based on large samples of hospital admissions. They found that the proportion of hospital admissions experiencing an adverse event, defined as causes of injuries by medical management, were 2.9 and $3.7\,$\%, respectively . The proportion of these events attributable to errors (that is, preventable adverse events, in other words medical errors) was 58 and $53\,$\%, respectively. Even when using the lower estimate, deaths due to preventable adverse events in 1997 in the US ($44\,000$) exceeded the deaths attributable to motor vehicle accidents ($43\,458$), breast cancer ($42\,297$) or AIDS ($16\,516$). Because many errors go unreported, detailed statistics on medical errors are fairly scarce (hospitals usually do not attempt to emphasize errors). Nevertheless, the problem is real. And this is not surprising, elaborate skills have to be mastered in the medical field. Moreover, achieving a high degree of personal competence is sometimes not sufficient as various members of staff must learn how to work as a team. In fact, one of the recommendations of this report \citep{Kohn00} was to \emph{establish interdisciplinary team training programs for providers that incorporate proven methods of training, such as simulation}. However, dispensing an appropriate training in such a complex and hazardous environment is very challenging. Because healthcare is a high risk industry (like aviation or military), training in the real world is too costly and dangerous. Consequently, various approaches were applied to teaching and training of medical practicians over history of healthcare. 

\bigskip

The most basic form of medical simulators are simple models of human anatomy. Hundreds of years ago, representations in clay and stone were already used to demonstrate clinical features of disease states and their effects on humans. Models have been found from many cultures and continents. Nowadays, similar passive models are still used to help students learn the anatomy. In constrast, active models that attempt to reproduce living anatomy or physiology are recent developments. The first example was created in the early 1960s by Asmund Laerdal for training in mouth to mouth ventilation. This simulator of a dying victim not breathing and lacking a heart beat, became known as called Resusci-Anne and had been widely used for CPR training thanks to an internal spring attached to the chest wall \citep{Cooper04}. In the mid-1960s, Sim One became the first mannequin controlled by a computer. The chest was moved with breathing, the eyes blinked and the pupils could dilate for instance. But the computer technology was too expensive for commercialisation at the time and only one mannequin was built. Over time mannequins have obviously substantially improved and today such systems are essentially integrated into training centers that aim at recreating the operating room environment. 


%- animal and fixed models used to reproduce some of the training objectives (but imperfect substitute and expensive)
%- animals, simulated patients, scenarios, computed-based



\section{Challenges for computer-based simulators}
%- Ultimate goal: needs to integrate patient-specific + physiology + pathologies for medical training, patient-specific planning and per-operative guidance		
%- Demands photorealistic rendering, accurate haptic feedback and (which is of prime interest to us): physically realist organ modelling in real-time.
%- Evolution: geometrical models => mass-springs => FEM => more elaborate constitutive laws / different type of structures
%- Contributions + plan
	
	
	
	
	