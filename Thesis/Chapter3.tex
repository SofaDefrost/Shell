\chapter{Practical approach of the Finite Element Method}
\label{chap3}
\begin{shortAbstract}
Blabla
\end{shortAbstract}


\section{Introduction}

	\subsection{A numerical method}
One of the most important things engineers and scientists do is to model physical phenomena. Using assumptions concerning how the phenomena works and using the appropriate laws of physics governing the process, they can derive a mathematical model, often characterised by complex algebraic, differential or integral equations relating various quantities of interest. However, because of the complexities in the geometry and complex boundary conditions found in real life problems, we cannot analytically solve these equations. A few decades ago, the only possible approach was to drastically simplify them, which was not always sufficient to find an approximate solution. Nowadays, in practice, most of the problems are solved using numerical methods. Indeed, with suitable mathematical models and numerical methods, computers can help solving many practical problems of engineering. Numerical methods typically transform differential equations governing a continuum to a set of algebraic equations of a discrete model of the continuum that are to be solved using computers \citep{Reddy93}. The \emph{Finite Element Method} (FEM) is the most popular numerical procedure that is used to approximately solve differential equations, especially in continuum mechanics. 
	
	\subsection{The basic ideas of FEM}
The finite element method begins by dividing the structure into small pieces, manageable regions, called \emph{elements}. The collection of all these elements makes up a \emph{mesh} which approximates the problem geometry. Why is this a good idea? If we can expect the solution for an engineering problem to be very complex, it is mostly due to the complex geometry, on which a global solution is difficult to figure out. If the problem domain can be divided (\emph{meshed}) into small elements, the solution within an element is easier to approximate \citep{MacDonald07}. Over each finite element, the unknown variables are approximated using known functions. These functions can be linear or higher-order polynomial expressions that depend on the geometrical locations (\emph{nodes}) used to define the finite element shape. The governing equations are integrated over each finite element using linear algebra techniques and the solution over the entire domain problem is obtained by summing (\emph{assembling}) the solution of each element. Thus, the finite element method transforms an infinite number of differential equations (one can be defined at any point of the continuum) into a finite number of algebraic equations (depending on the chosen number of elements). 

This approach may be compared to trying to find the area under a curve. We know that we can find the exact solution for the area under the curve by integration. However, sometimes the function describing the curve is not known, or is difficult to integrate. One method to obtain an approximate solution is to break up the area into a series of rectangles and add the areas of all rectangles \TODO{Add figure with curve and rectangles}. It is worth nothing that the solution accuracy can be increased by reducing the width of the rectangles to better follow the curve. 

\bigskip

It is crucial to keep in mind that approximations occur at different stages during finite element analysis. The division of the whole domain into finite elements may not be exact \TODO{add figure from a domain divided into elements}, introducing error in the domain being modelled. The second stage is when element equations are derived. As mentionned earlier, the unknowns of the problem are approximated using the idea that any continuous function can be representated by a linear combination of known functions and undetermined coefficients. Algebraic relations between the undetermined coefficients are then obtained by satisfying the governing equations over each element. There are a few types of approach for establishing these equations but, without going into details, the mathematical foundation of all these approaches is the \emph{weighted residual method} (see Appendix~\ref{appendix2}), which leads to integrals during the process. Therefore, the second stage creates two sources of error: the representation of the solution by a linear combination of functions and the evaluation of the integrals. Finally, errors are introduced when solving the assembled system of algebraic equations. 


\section{Discretisation}

	\subsection{Meshing process}
The first step in the finite element method is to create a mesh of the domain to study. Mesh generation is a very important task and can be very time consuming. The domain has to be meshed properly into elements of specific shapes. All the elements together form the entire domain of the problem without any gap or overlapping. For example, triangles or quadrilaterals can be used in two dimensions, and tetrahedra and hexahedra in three dimensions. Information, such as \emph{element connectivity}, must be created during the meshing process for later use in the formulation of the FEM equations. The number of elements into which the domain is divided in a problem depends mainly on the geometry of the domain and on the desired accuracy of the solution. Usually, the number of elements increases with the complexity of the geometry. For instance, if one part of the domain is thiner \TODO{add figure to illustrate this point}, the size of the elements must be reduced in order to tile this particular part, which increases the total number of elements, and hence  the number of algebraic equations to solve. However, adding elements is sometimes desirable. Indeed, increasing the number of elements tends to get an approximate solution closer to the exact analytical solution (as reducing the width of the rectangles allows for a better approximation of the area under the curve). A trade-off between accuracy and computational time must be found. This is why the meshing process on the problem domain must be carried out carefully. One needs to create a mesh which gives an accurate enough solution for the desired application while restraining the computational time according to the problem time constraints. As an example, finding the maximum load that can sustain a bridge in structural mechanics demands a high degree of accuracy, no matter how much time it is needed to compute the solution. Conversely, organ deformation in medical training simulators must be computed at an interactive rate so that no apparent delay can be observed between the manipulation of a given organ and its deformation. It does not mean that precision is not required, simply that the time constraints will necessarily limit the accuracy of the solution. 

	
	\subsection{Elements and shape functions}
As we have seen, the finite element method is based on finding an approximate solution over each element rather than the whole domain. To connect the approximate solution from each element to form a continuous over the whole domain, we require the solution to be the same at points common to the elements. Therefore, the element nodes (vertices defining each element's shape) needs to play the role of interpolation points in constructing the approximation functions over an element. Depending on the degree of polynomial approximation used to represent the solution, additional nodes may be identified inside the element. It is worth noting that the type of interpolation directly affects the accuracy of the solution. 

	
	
\section{Derivation of element equations}	
	\subsection{Numerical integration}
	\subsection{Generation of element matrices}

\section{Assembly of element equations}

\section{Solution of global problem}
		



